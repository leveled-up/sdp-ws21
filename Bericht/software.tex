\section{Softwarekonzept}
% Beschreiben Sie hier, wie Sie die Programmierung des Roboters umgesetzt haben und erklären Sie im Detail z.B. die State Machine und den PID-Regler.

Das Softwarekonzept des Roboters zeichnet sich durch eine klare Trennung verschiedener Abläufe aus, deren Sequenz in \texttt{task main()} definiert ist. Nach Initialisierung besteht das Programm aus den Grundoperationen \texttt{readSensors()} zum Auslesen, Interpretieren und Speichern der Sensorwerte, \texttt{show()} zur Ausgabe des Zustands und der Sensorwerte auf dem Display und \textit{Engine Control}, Funktionen zur Steuerung der Motoren. Diese Operationen laufen im \textit{\enquote{Course correction loop}} zusammen, einer Endlosschleife die den aktuellen Stand des Roboters mit \texttt{readSensors()} (und dem Zähler \texttt{cyclesNoLine}) ermittelt und mit den Funktionen des \textit{Engine Control} daraus Konsequenzen für die Fahrt des Robotes ableitet, die das Gerät dann für einen \textit{\enquote{Zyklus}} (\texttt{TTL}) beibehält. Wird das Ende der Strecke erkannt, bricht die Schleife ab und das Programm endet.